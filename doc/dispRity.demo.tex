\documentclass{article}\usepackage[]{graphicx}\usepackage[]{color}
%% maxwidth is the original width if it is less than linewidth
%% otherwise use linewidth (to make sure the graphics do not exceed the margin)
\makeatletter
\def\maxwidth{ %
  \ifdim\Gin@nat@width>\linewidth
    \linewidth
  \else
    \Gin@nat@width
  \fi
}
\makeatother

\definecolor{fgcolor}{rgb}{0.345, 0.345, 0.345}
\newcommand{\hlnum}[1]{\textcolor[rgb]{0.686,0.059,0.569}{#1}}%
\newcommand{\hlstr}[1]{\textcolor[rgb]{0.192,0.494,0.8}{#1}}%
\newcommand{\hlcom}[1]{\textcolor[rgb]{0.678,0.584,0.686}{\textit{#1}}}%
\newcommand{\hlopt}[1]{\textcolor[rgb]{0,0,0}{#1}}%
\newcommand{\hlstd}[1]{\textcolor[rgb]{0.345,0.345,0.345}{#1}}%
\newcommand{\hlkwa}[1]{\textcolor[rgb]{0.161,0.373,0.58}{\textbf{#1}}}%
\newcommand{\hlkwb}[1]{\textcolor[rgb]{0.69,0.353,0.396}{#1}}%
\newcommand{\hlkwc}[1]{\textcolor[rgb]{0.333,0.667,0.333}{#1}}%
\newcommand{\hlkwd}[1]{\textcolor[rgb]{0.737,0.353,0.396}{\textbf{#1}}}%

\usepackage{framed}
\makeatletter
\newenvironment{kframe}{%
 \def\at@end@of@kframe{}%
 \ifinner\ifhmode%
  \def\at@end@of@kframe{\end{minipage}}%
  \begin{minipage}{\columnwidth}%
 \fi\fi%
 \def\FrameCommand##1{\hskip\@totalleftmargin \hskip-\fboxsep
 \colorbox{shadecolor}{##1}\hskip-\fboxsep
     % There is no \\@totalrightmargin, so:
     \hskip-\linewidth \hskip-\@totalleftmargin \hskip\columnwidth}%
 \MakeFramed {\advance\hsize-\width
   \@totalleftmargin\z@ \linewidth\hsize
   \@setminipage}}%
 {\par\unskip\endMakeFramed%
 \at@end@of@kframe}
\makeatother

\definecolor{shadecolor}{rgb}{.97, .97, .97}
\definecolor{messagecolor}{rgb}{0, 0, 0}
\definecolor{warningcolor}{rgb}{1, 0, 1}
\definecolor{errorcolor}{rgb}{1, 0, 0}
\newenvironment{knitrout}{}{} % an empty environment to be redefined in TeX

\usepackage{alltt}
\usepackage[sc]{mathpazo}
\usepackage[T1]{fontenc}
\usepackage{geometry}
\geometry{verbose,tmargin=2.5cm,bmargin=2.5cm,lmargin=2.5cm,rmargin=2.5cm}
\setcounter{secnumdepth}{2}
\setcounter{tocdepth}{2}
\usepackage{url}
\usepackage[unicode=true,pdfusetitle,
 bookmarks=true,bookmarksnumbered=true,bookmarksopen=true,bookmarksopenlevel=2,
 breaklinks=false,pdfborder={0 0 1},backref=false,colorlinks=false]
 {hyperref}
\hypersetup{
 pdfstartview={XYZ null null 1}}

\newcommand{\dispRity}{\texttt{dispRity} }
\newcommand{\R}{\texttt{R} }
\IfFileExists{upquote.sty}{\usepackage{upquote}}{}
\begin{document}



\title{dispRity demo}


\author{Thomas Guillerme}

\maketitle

This a quick demo to go through the beta version of the \dispRity package (v.0.1.0).
Many parts are still missing (see section \ref{whatsleft}) but this version should be running all right and give an idea of what's implemented in the package.

\section{Before starting}

\dispRity is a package for calculating disparity in \R.
What is disparity? Well that's a more complex question...
To keep it short, this package allows to summarise ordinated matrices (e.g. MDS, PCA, PCO, PCoA) into single values.
% More explanations to put here?

\subsection{Installation}
You can install this package pretty easily if you use the latest version of \R and \texttt{devtools}.
Just copy past the following
\begin{knitrout}
\definecolor{shadecolor}{rgb}{0.969, 0.969, 0.969}\color{fgcolor}\begin{kframe}
\begin{alltt}
\hlkwd{install.packages}\hlstd{(}\hlstr{"devtools"}\hlstd{)}
\hlkwd{library}\hlstd{(devtools)}
\hlkwd{install_github}\hlstd{(}\hlstr{"TGuillerme/dispRity"}\hlstd{,} \hlkwc{ref}\hlstd{=}\hlstr{"release"}\hlstd{)}
\hlkwd{library}\hlstd{(dispRity)}
\end{alltt}
\end{kframe}
\end{knitrout}

Note that we use here the \texttt{release} branch which is version 0.1.0.
If you want the piping-hot (and full of bugs) version, change the option \texttt{ref="release"} to \texttt{ref="master"}.

This package depends heavily on the \texttt{ape} package as well as the nice function \texttt{timeSliceTree::paleotree}.
However, I recommend to also install the packages \texttt{geomorph} or \texttt{Claddis} if you want to try this example on your own morphometric or cladistic data.

\subsection{Which data?}
You'll need to do this and that
%To develop

\begin{knitrout}
\definecolor{shadecolor}{rgb}{0.969, 0.969, 0.969}\color{fgcolor}\begin{kframe}
\begin{alltt}
\hlcom{## Loading demo and the package data}
\hlkwd{library}\hlstd{(dispRity)}
\end{alltt}


{\ttfamily\noindent\itshape\color{messagecolor}{\#\# Loading required package: paleotree}}\begin{alltt}
\hlcom{## An ordinated matrix with the tips only}
\hlkwd{data}\hlstd{(BeckLee_mat50)}
\hlkwd{dim}\hlstd{(BeckLee_mat50)}
\end{alltt}
\begin{verbatim}
## [1] 50 48
\end{verbatim}
\begin{alltt}
\hlkwd{head}\hlstd{(BeckLee_mat50[,}\hlnum{1}\hlopt{:}\hlnum{5}\hlstd{])}
\end{alltt}
\begin{verbatim}
##                    [,1]          [,2]        [,3]       [,4]      [,5]
## Cimolestes   -0.5319679  0.1117759259  0.09865194 -0.1933148 0.2035833
## Maelestes    -0.4087147  0.0139690317  0.26268300  0.2297096 0.1310953
## Batodon      -0.6923194  0.3308625215 -0.10175223 -0.1899656 0.1003108
## Bulaklestes  -0.6802291 -0.0134872777  0.11018009 -0.4103588 0.4326298
## Daulestes    -0.7386111  0.0009001369  0.12006449 -0.4978191 0.4741342
## Uchkudukodon -0.5105254 -0.2420633915  0.44170317 -0.1172972 0.3602273
\end{verbatim}
\begin{alltt}
\hlcom{## An ordinated matrix with the tips and the nodes (named nX)}
\hlkwd{data}\hlstd{(BeckLee_mat99)}
\hlkwd{dim}\hlstd{(BeckLee_mat99)}
\end{alltt}
\begin{verbatim}
## [1] 99 97
\end{verbatim}
\begin{alltt}
\hlkwd{head}\hlstd{(BeckLee_mat99[,}\hlnum{1}\hlopt{:}\hlnum{5}\hlstd{],} \hlnum{2}\hlstd{)}
\end{alltt}
\begin{verbatim}
##                  [,1]       [,2]       [,3]       [,4]        [,5]
## Cimolestes -0.6082437 -0.0323683 0.08458885 -0.4338448 -0.30536875
## Maelestes  -0.5730206 -0.2840361 0.01308847 -0.1258848  0.06123611
\end{verbatim}
\begin{alltt}
\hlkwd{tail}\hlstd{(BeckLee_mat99[,}\hlnum{1}\hlopt{:}\hlnum{5}\hlstd{],} \hlnum{2}\hlstd{)}
\end{alltt}
\begin{verbatim}
##            [,1]      [,2]       [,3]       [,4]       [,5]
## n48 -0.05529018 0.4799330 0.04118477 0.04944912 -0.3558830
## n49 -0.13067785 0.4478168 0.11956268 0.13800340 -0.3222785
\end{verbatim}
\begin{alltt}
\hlcom{## A list of first and last occurence data for some fossils}
\hlkwd{data}\hlstd{(BeckLee_ages)}
\hlkwd{head}\hlstd{(BeckLee_ages)}
\end{alltt}
\begin{verbatim}
##             FAD  LAD
## Adapis     37.2 36.8
## Asioryctes 83.6 72.1
## Leptictis  33.9 33.3
## Miacis     49.0 46.7
## Mimotona   61.6 59.2
## Notharctus 50.2 47.0
\end{verbatim}
\begin{alltt}
\hlcom{## And a phylogeny:}
\hlkwd{data}\hlstd{(BeckLee_tree)}
\hlkwd{plot}\hlstd{(BeckLee_tree) ;} \hlkwd{nodelabels}\hlstd{(}\hlkwc{cex}\hlstd{=}\hlnum{0.8}\hlstd{)}
\end{alltt}
\end{kframe}

{\centering \includegraphics[width=.8\linewidth]{figure/minimal-data-1} 

}



\end{knitrout}

\section{A super quick go through}
This package allows to:
\begin{enumerate}
    \item Splitting the data
    \item Bootstrapping the data
    \item Calculating disparity
    \item Summarising the results
\end{enumerate}

\begin{knitrout}
\definecolor{shadecolor}{rgb}{0.969, 0.969, 0.969}\color{fgcolor}\begin{kframe}
\begin{alltt}
\hlcom{## Splitting the data}
\hlstd{sliced_data} \hlkwb{<-} \hlkwd{time.series}\hlstd{(BeckLee_mat99, BeckLee_tree,} \hlkwc{method} \hlstd{=} \hlstr{"continuous"}\hlstd{,}
    \hlkwc{model} \hlstd{=} \hlstr{"acctran"}\hlstd{,} \hlkwc{time} \hlstd{=} \hlnum{5}\hlstd{,} \hlkwc{FADLAD} \hlstd{= BeckLee_ages)}
\end{alltt}


{\ttfamily\noindent\itshape\color{messagecolor}{\#\# Some tips have FAD/LAD and are assumed to interval single points in time.}}\begin{alltt}
\hlcom{## Bootstrapping the data}
\hlstd{bootstrapped_data} \hlkwb{<-} \hlkwd{boot.matrix}\hlstd{(sliced_data,} \hlnum{100}\hlstd{)}

\hlcom{## Calculating disparity}
\hlstd{sum_of_ranges} \hlkwb{<-} \hlkwd{dispRity}\hlstd{(bootstrapped_data,} \hlkwc{metric} \hlstd{=} \hlkwd{c}\hlstd{(sum, range))}

\hlcom{## Summarising the results}
\hlkwd{summary}\hlstd{(sum_of_ranges)}
\end{alltt}
\begin{verbatim}
##      series  n   mean   2.5%    25%    75%  97.5%
## 1 133.51104  3 -0.348 -0.397 -0.350 -0.349 -0.230
## 2 100.13328 14  0.101 -0.252 -0.046  0.214  0.291
## 3  66.75552 23 -0.059 -0.163 -0.112  0.031  0.115
## 4  33.37776 11  0.008 -0.134 -0.086  0.132  0.217
## 5         0 10  0.009 -0.134 -0.086  0.140  0.187
\end{verbatim}
\begin{alltt}
\hlkwd{plot}\hlstd{(sum_of_ranges,} \hlkwc{type} \hlstd{=} \hlstr{"continuous"}\hlstd{)}
\end{alltt}
\end{kframe}

{\centering \includegraphics[width=.8\linewidth]{figure/minimal-Quick_go-1} 

}



\end{knitrout}


\section{In more details}

\subsection{The \dispRity objects}
Disparity analysis involve a lot of shuffling around with many matrices (especially when bootstrapping the data) which can be a bit impractical to visualise and quickly jam your \R console.
For example, try to use \texttt{str} to show the structure of the object created above:

\begin{knitrout}
\definecolor{shadecolor}{rgb}{0.969, 0.969, 0.969}\color{fgcolor}\begin{kframe}
\begin{alltt}
\hlkwd{str}\hlstd{(sum_of_ranges)}
\end{alltt}
\end{kframe}
\end{knitrout}

Therefore this package proposes a new class of object called \dispRity (how unexpected!) that allow to use a \texttt{S3 print.dispRity} method.
In other words, when you are creating objects using the \dispRity functions, you can quickly and easily have a look at them by just printing it.
For example:

\begin{knitrout}
\definecolor{shadecolor}{rgb}{0.969, 0.969, 0.969}\color{fgcolor}\begin{kframe}
\begin{alltt}
\hlcom{## Which class is the sum_of_ranges object?}
\hlkwd{class}\hlstd{(sum_of_ranges)}
\end{alltt}
\begin{verbatim}
## [1] "dispRity"
\end{verbatim}
\begin{alltt}
\hlcom{## We can summarise it using the S3 method print.dispRity}
\hlstd{sum_of_ranges}
\end{alltt}
\begin{verbatim}
## Disparity measurements across 5 series for 99 taxa. 
## Series:
## 133.51104, 100.13328, 66.75552, 33.37776, 0.
## Disparity calculated as: sum range for 97 dimensions.
## Data was split using continuous method.
## Data was bootstrapped 100 times, using the full bootstrap method.
\end{verbatim}
\end{kframe}
\end{knitrout}

Note that each object output from different functions will show summarise slightly different informations.

\subsection{Splitting the data}
One of the first functionality of this package is to facilitate splitting the ordinated matrix in subsamples.
The original ordinated matrix represents the total any-o-space.

\subsubsection{Custom splitting}
\texttt{cust.series} is a fairly straightforward function and allows to split the data according to a factor determined by the user.
For example, here we can split the matrix based on phylogeny: let's classify in two different groups the crown and stem mammals present in the tree.

\begin{knitrout}
\definecolor{shadecolor}{rgb}{0.969, 0.969, 0.969}\color{fgcolor}\begin{kframe}
\begin{alltt}
\hlcom{## We want to separate the species around the node 71 (see phylogeny above).}
\hlcom{## All the descendant are crown and all the ancestors are stem}
\hlstd{crown} \hlkwb{<-} \hlkwd{extract.clade}\hlstd{(BeckLee_tree,} \hlkwc{node} \hlstd{=} \hlnum{71}\hlstd{)}\hlopt{$}\hlstd{tip.label}
\hlstd{stem} \hlkwb{<-} \hlkwd{drop.tip}\hlstd{(BeckLee_tree,} \hlkwc{tip} \hlstd{= crown)}\hlopt{$}\hlstd{tip.label}

\hlcom{## We then have to feed this information in a data.frame with one column}
\hlstd{factors} \hlkwb{<-} \hlkwd{as.data.frame}\hlstd{(}
    \hlkwd{matrix}\hlstd{(}\hlkwc{data} \hlstd{=} \hlkwd{c}\hlstd{(}\hlkwd{rep}\hlstd{(}\hlstr{"crown"}\hlstd{,} \hlkwd{length}\hlstd{(crown)),} \hlkwd{rep}\hlstd{(}\hlstr{"stem"}\hlstd{,} \hlkwd{length}\hlstd{(stem))),}
    \hlkwc{ncol} \hlstd{=} \hlnum{1}\hlstd{,} \hlkwc{dimnames} \hlstd{=} \hlkwd{list}\hlstd{(}\hlkwd{c}\hlstd{(crown, stem))))}

\hlcom{## Then we can use the customised series function}
\hlstd{crown_stem} \hlkwb{<-} \hlkwd{cust.series}\hlstd{(BeckLee_mat50, factors)}

\hlcom{## This created a dispRity object containing two series: crown and stem}
\hlkwd{class}\hlstd{(crown_stem)}
\end{alltt}
\begin{verbatim}
## [1] "dispRity"
\end{verbatim}
\begin{alltt}
\hlstd{crown_stem}
\end{alltt}
\begin{verbatim}
## 2 custom series for 50 taxa. 
## Series:
## crown, stem.
\end{verbatim}
\begin{alltt}
\hlcom{## This object contains three elements}
\hlkwd{names}\hlstd{(crown_stem)}
\end{alltt}
\begin{verbatim}
## [1] "data"   "taxa"   "series"
\end{verbatim}
\begin{alltt}
\hlcom{## With "data" being the list sub-matrices}
\hlkwd{str}\hlstd{(crown_stem}\hlopt{$}\hlstd{data)}
\end{alltt}
\begin{verbatim}
## List of 2
##  $ crown: num [1:30, 1:48] 0.3079 0.6531 0.5089 -0.1652 -0.0419 ...
##   ..- attr(*, "dimnames")=List of 2
##   .. ..$ : chr [1:30] "Dasypodidae" "Bradypus" "Myrmecophagidae" "Todralestes" ...
##   .. ..$ : NULL
##  $ stem : num [1:20, 1:48] -0.739 -0.68 -0.511 -0.477 -0.473 ...
##   ..- attr(*, "dimnames")=List of 2
##   .. ..$ : chr [1:20] "Daulestes" "Bulaklestes" "Uchkudukodon" "Kennalestes" ...
##   .. ..$ : NULL
\end{verbatim}
\begin{alltt}
\hlcom{## "taxa" being the list of taxa in the original ordinated matrix}
\hlkwd{str}\hlstd{(crown_stem}\hlopt{$}\hlstd{taxa)}
\end{alltt}
\begin{verbatim}
##  chr [1:50] "Cimolestes" "Maelestes" "Batodon" "Bulaklestes" "Daulestes" ...
\end{verbatim}
\begin{alltt}
\hlcom{## "series" containing information on the series type (custom) and names (crown and stem)}
\hlstd{crown_stem}\hlopt{$}\hlstd{series}
\end{alltt}
\begin{verbatim}
## [1] "custom" "crown"  "stem"
\end{verbatim}
\end{kframe}
\end{knitrout}

Of course, using phylogeny as a factor is just an example.
Any type of factors can be used in this example (e.g. body mass categories, habitat, diet, etc...).
Which leads us to another type of category used a lot in disparity-through-time analysis: time!

\subsubsection{Time splitting}
\texttt{time.series} is a function that allows to split the matrix in different ways using time as a category.
Two types of time series splitting can be performed with the \dispRity package by using the \texttt{method} option in \texttt{time.series}.
\begin{enumerate}
\item the classical discrete time splitting (or time-binning) using \texttt{method = "discrete"};
\item or a new continuous time splitting (or time-slicing) using \texttt{method = "continuous"}.
\end{enumerate}


\subsection{Bootstrapping the data}

\subsection{Calculating disparity}

\subsection{Summarising the results}

\section{What's left?}
\label{whatsleft}

\end{document}
