\documentclass{article}\usepackage[]{graphicx}\usepackage[]{color}
%% maxwidth is the original width if it is less than linewidth
%% otherwise use linewidth (to make sure the graphics do not exceed the margin)
\makeatletter
\def\maxwidth{ %
  \ifdim\Gin@nat@width>\linewidth
    \linewidth
  \else
    \Gin@nat@width
  \fi
}
\makeatother

\definecolor{fgcolor}{rgb}{0.345, 0.345, 0.345}
\newcommand{\hlnum}[1]{\textcolor[rgb]{0.686,0.059,0.569}{#1}}%
\newcommand{\hlstr}[1]{\textcolor[rgb]{0.192,0.494,0.8}{#1}}%
\newcommand{\hlcom}[1]{\textcolor[rgb]{0.678,0.584,0.686}{\textit{#1}}}%
\newcommand{\hlopt}[1]{\textcolor[rgb]{0,0,0}{#1}}%
\newcommand{\hlstd}[1]{\textcolor[rgb]{0.345,0.345,0.345}{#1}}%
\newcommand{\hlkwa}[1]{\textcolor[rgb]{0.161,0.373,0.58}{\textbf{#1}}}%
\newcommand{\hlkwb}[1]{\textcolor[rgb]{0.69,0.353,0.396}{#1}}%
\newcommand{\hlkwc}[1]{\textcolor[rgb]{0.333,0.667,0.333}{#1}}%
\newcommand{\hlkwd}[1]{\textcolor[rgb]{0.737,0.353,0.396}{\textbf{#1}}}%

\usepackage{framed}
\makeatletter
\newenvironment{kframe}{%
 \def\at@end@of@kframe{}%
 \ifinner\ifhmode%
  \def\at@end@of@kframe{\end{minipage}}%
  \begin{minipage}{\columnwidth}%
 \fi\fi%
 \def\FrameCommand##1{\hskip\@totalleftmargin \hskip-\fboxsep
 \colorbox{shadecolor}{##1}\hskip-\fboxsep
     % There is no \\@totalrightmargin, so:
     \hskip-\linewidth \hskip-\@totalleftmargin \hskip\columnwidth}%
 \MakeFramed {\advance\hsize-\width
   \@totalleftmargin\z@ \linewidth\hsize
   \@setminipage}}%
 {\par\unskip\endMakeFramed%
 \at@end@of@kframe}
\makeatother

\definecolor{shadecolor}{rgb}{.97, .97, .97}
\definecolor{messagecolor}{rgb}{0, 0, 0}
\definecolor{warningcolor}{rgb}{1, 0, 1}
\definecolor{errorcolor}{rgb}{1, 0, 0}
\newenvironment{knitrout}{}{} % an empty environment to be redefined in TeX

\usepackage{alltt}
\usepackage[sc]{mathpazo}
\usepackage[T1]{fontenc}
\usepackage{geometry}
\geometry{verbose,tmargin=2.5cm,bmargin=2.5cm,lmargin=2.5cm,rmargin=2.5cm}
\setcounter{secnumdepth}{2}
\setcounter{tocdepth}{2}
\usepackage{url}
\usepackage[unicode=true,pdfusetitle,
 bookmarks=true,bookmarksnumbered=true,bookmarksopen=true,bookmarksopenlevel=2,
 breaklinks=false,pdfborder={0 0 1},backref=false,colorlinks=false]
 {hyperref}
\hypersetup{
 pdfstartview={XYZ null null 1}}

\newcommand{\dispRity}{\texttt{dispRity} }
\newcommand{\R}{\texttt{R} }
\IfFileExists{upquote.sty}{\usepackage{upquote}}{}
\begin{document}



\title{dispRity demo}


\author{Thomas Guillerme}

\maketitle

This a quick demo to go through the beta version of the \dispRity package (v.0.1.0).
Many parts are still missing (see section \ref{whatsleft}) but this version should be running all right and give an idea of what's implemented in the package.

\section{Before starting}

\dispRity is a package for calculating disparity in \R.
What is disparity? Well that's a more complex question...
To keep it short, this package allows to summarise ordinated matrices (e.g. MDS, PCA, PCO, PCoA) into single values.
% More explanations to put here?

\subsection{Installation}
You can install this package pretty easily if you use the latest version of \R and \texttt{devtools}.
Just copy past the following
\begin{knitrout}
\definecolor{shadecolor}{rgb}{0.969, 0.969, 0.969}\color{fgcolor}\begin{kframe}
\begin{alltt}
\hlkwd{install.packages}\hlstd{(}\hlstr{"devtools"}\hlstd{)}
\hlkwd{library}\hlstd{(devtools)}
\hlkwd{install_github}\hlstd{(}\hlstr{"TGuillerme/dispRity"}\hlstd{,} \hlkwc{ref}\hlstd{=}\hlstr{"release"}\hlstd{)}
\hlkwd{library}\hlstd{(dispRity)}
\end{alltt}
\end{kframe}
\end{knitrout}

Note that we use here the \texttt{release} branch which is version 0.1.0.
If you want the piping-hot (and full of bugs) version, change the option \texttt{ref="release"} to \texttt{ref="master"}.

This package depends heavily on the \texttt{ape} package as well as the nice function \texttt{timeSliceTree::paleotree}.
However, I recommend to also install the packages \texttt{geomorph} or \texttt{Claddis} if you want to try this example on your own morphometric or cladistic data.

\subsection{Getting the data}
You'll need to do this and that

\section{A super quick go through}
This package allows to:
\begin{enumerate}
    \item Separate your ordinated data into different series for example
    \item Bootstrap this data
    \item Calculate a disparity metric from it
    \item Summarise and plot the disparity
\end{enumerate}

\begin{knitrout}
\definecolor{shadecolor}{rgb}{0.969, 0.969, 0.969}\color{fgcolor}\begin{kframe}
\begin{alltt}
\hlcom{## Separate your ordinated data into different series for example}
\hlnum{1}\hlopt{+}\hlnum{1}
\end{alltt}
\begin{verbatim}
## [1] 2
\end{verbatim}
\begin{alltt}
\hlcom{## Bootstrap this data}
\hlnum{1}\hlopt{+}\hlnum{1}
\end{alltt}
\begin{verbatim}
## [1] 2
\end{verbatim}
\begin{alltt}
\hlcom{## Calculate a disparity metric from it}
\hlnum{1}\hlopt{+}\hlnum{1}
\end{alltt}
\begin{verbatim}
## [1] 2
\end{verbatim}
\begin{alltt}
\hlcom{# Summarise and plot the disparity}
\hlkwd{summary}\hlstd{(}\hlnum{1}\hlstd{)}
\end{alltt}
\begin{verbatim}
##    Min. 1st Qu.  Median    Mean 3rd Qu.    Max. 
##       1       1       1       1       1       1
\end{verbatim}
\begin{alltt}
\hlkwd{plot}\hlstd{(}\hlnum{1}\hlstd{)}
\end{alltt}
\end{kframe}

{\centering \includegraphics[width=.4\linewidth]{figure/minimal-Quick_go-1} 

}



\end{knitrout}

\section{What's left?}
\label{whatsleft}

\end{document}
